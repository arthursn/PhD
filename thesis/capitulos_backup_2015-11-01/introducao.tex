\chapter{Introdu\c{c}\~{a}o}

Ferros fundidos consistem de uma das primeiras classes de produtos ferrosos utilizados pela civilização. Atualmente, ainda possuem grande importância tecnológica, em virtude de seu baixo custo de produção, condutividade térmica, usinabilidade e excelente capacidade de absorção de ondas mecânicas decorrentes da presença de grafita. No entanto, a mesma grafita que propicia estas propriedades também é responsável por diminuir a ductilidade, a resistência mecânica e à fadiga do material. Estratégias para contornar esses problemas incluem a modificação da morfologia de grafita --- grafita nodular gera um fator de concentração de tensões significativamente inferior àquele que acompanha a grafita lamelar --- e tratamentos térmicos para obtenção de matrizes que forneçam resistência mecânica e/ou ductilidade superior.

Ferros fundidos nodulares, por terem sua microestrutura composta de grafita nodular, têm a ductilidade uma propriedade importante do material. Uma notável família de ferros fundidos nodulares é a dos austemperados (ADI, abreviatura de \textit{Austempered Ductile Iron}). Estes nodulares apresentam uma matriz de ausferrita, i.e., uma mistura de ferrita bainítica e austenita retida, que proporciona bons valores de limites de resistência (entre 850 a 1300 MPa) e ductilidade (entre 2 e 10\%)\cite{Guesser2009}. Em decorrência destas excelentes propriedades, aliadas aos baixos custos de produção, componentes de ADI foram rapidamente agregados à indústria, notavelmente a automotiva, sendo competitivos com produtos forjados\cite{Hayrynen2002}.

O relativo sucesso do ADI veio com o recente desenvolvimento de ligas ferrosas contendo significativas quantidades de austenita retida. Pickering na década de 1970 sugeriu que microestruturas bifásicas contendo ferrita e austenita retida poderiam ser produzidas pelo interrompimento da decomposição eutetóide da austenita, sendo úteis no projeto de ligas com elevadas resistência e tenacidade\cite{Goldenstein2002}. Além de aproveitada pela indústria de peças fundidas, esta ideia foi retomada por vários pesquisadores, sendo notáveis os trabalhos de Bhadeshia, Edmonds e Caballero\cite{Bhadeshia1980a,Caballero2003,Garcia-Mateo2009,GarciaMateo2005}. Contrariando a nomenclatura utilizada em fundição, estes autores optaram por denominar este agregado ferrítico-austenítico em acordo com o produto da reação bainítica, isto é,\textit{ carbide-free bainite}, ou bainita isenta de carbonetos.

Mais recentemente, \citaremsentenca{Speer2003} propuseram uma nova rota de tratamento térmico denominada \textquotedblleft{}Têmpera e Partição\textquotedblright{} (T\&P) para obtenção de uma microestrutura constituída de uma mistura de martensita em equilíbrio metaestável com austenita. O conceito deste processo envolve a têmpera parcial da austenita em temperaturas intermediárias ao Ms e ao Mf, seguido de um tratamento térmico de partição, para que ocorra a redistribuição do carbono da martensita para a austenita. Uma vez que a austenita é tão mais estável quanto mais enriquecida em elementos de liga, o tratamento de partição visa a estabilização desta fase mesmo à temperatura ambiente. No entanto, este processo somente é possível caso as reações do revenido, dentre as quais a precipitação de carbonetos é a principal, sejam suprimidas suficientemente para que haja o enriquecimento da austenita em carbono. Isto é conseguido pela adição de elementos como o Si e o Al na liga.

Speer também sugeriu que o processo T\&P poderia ser utilizado na substituição do processo de produção de ADI. Os resultados dos poucos trabalhos publicados sobre este assunto se mostraram promissores, embora difíceis de serem interpretados\cite{Speer2004}. Resultados recentes de trabalhos conduzidos pelo grupo de pesquisa em que se insere o presente projeto sugeriram que a microsegregação inerente aos ferros fundidos poderia justificar a dificuldade reportada. Contudo, estas evidências, de significativa importância científica e tecnológica, carecem de caracterização e análise minuciosa\cite{Anderson2013}. Face a este desafio, este trabalho busca a melhor compreensão do fenômeno que circunda o processo de Têmpera e Partição para o especifico caso de ferros fundidos nodulares.

%A maioria das transformações de fases do sistema Fe-C envolve a difusão de carbono e são relativamente bem compreendidos, embora alguns aspectos sobre o mecanismo que acompanha estas reações ainda são motivo de discussão e controvérsia, sendo notável o caso do debate sobre a bainita. Em virtude dos menores interstícios existentes, a ferrita possui menor solubilidade em carbono do que a austenita\cite{Honeycombe2006}. Por este motivo, a formação da ferrita promove o enriquecimento local da austenita. Recentes desenvolvimentos em aços bainíticos ligados com elementos supressores de carbonetos fazem uso deste fenômeno para produzir microestruturas constituídas de finas ripas de ferrita bainítica e austenita estabilizada\cite{Bhadeshia1980,Caballero2003,Garcia-Mateo2009,GarciaMateo2005,Caballero2001,Chang2004,Yakubtsov2011}. %Isso vai pra introdução

%Por sua vez, a formação da martensita ocorre por um mecanismo ``displacivo'' e ocorre sem a difusão de carbono, fato que permite que a nova fase se forme supersaturada em carbono. Até recentemente, o papel da difusão do carbono no enriquecimento da austenita em tratamentos posteriores à têmpera fora pouco abordado, embora este fenômeno seja conhecido desde a década de 1960\cite{Matas1961}. Utilizando desde conceito, Speer e colaboradores\cite{Speer2003} propuseram uma nova rota para produção de aços de elevada resistência, por meio da obtenção de microestruturas compostas de quantidades controladas de martensita e austenita estabilizada por carbono. Esta rota, que consiste de um tratamento térmico em uma ou duas etapas, é denominado Têmpera e Partição (do inglês \textit{Quenching and Partitioning}, ou T\&P) e é esquematicamente representado na figura \ref{esqTP}. A martensita confere boa resistência mecânica, enquanto a austenita estabilizada promove ductilidade devido ao efeito de plasticidade induzida por deformação (efeito TRIP). %Isso também vai pra introdução
