\chapter{Introdução}

Embora o ferro fundido seja frequentemente considerado um material antiquado --- provavelmente devido à comparação com o aço, seu parente mais nobre --- esta classe de ligas ferrosas ainda hoje possui relevante importância tecnológica. Além de seu baixo custo de produção, ferros fundidos possuem características únicas decorrentes da presença de grafita, como elevada condutividade térmica, boa usinabilidade e excelente capacidade de absorção de ondas mecânicas. No entanto, a mesma grafita que propicia estas propriedades também é responsável por diminuir a ductilidade, a resistência mecânica e a resistência à fadiga do material. Estratégias para contornar esses problemas incluem a modificação da morfologia de grafita e tratamentos térmicos para obtenção de matrizes que forneçam resistência mecânica e/ou ductilidade superior.

Ferros fundidos nodulares consistem de ferros fundidos em que a morfologia predominante de grafita é na forma de nódulos esféricos. A grafita nodular gera um fator de concentração de tensões significativamente inferior do que a grafita lamelar, o que confere ao material elevada ductilidade quando comparada a outras classes de ferros fundidos. Devido a esta característica, a terminologia mais comum do ferro fundido nodular no inglês é \textit{Ductile Cast Iron}, isto é, ferro fundido dúctil. Existem também distinções dentro da classe de ferros fundidos nodulares em relação aos diferentes tipos de microestruturas presentes na matriz do ferro fundido. Um importante exemplo são os ferros fundidos nodulares austemperados (ADI, abreviatura de \textit{Austempered Ductile Iron}). Estes nodulares apresentam uma matriz de ausferrita, microconstituinte que consiste da mistura de ferrita bainítica e austenita retida, que proporciona bons valores de limites de resistência (entre 850 a 1300 MPa) e ductilidade (entre 2 e 10\%)\cite{Guesser2009}. Em decorrência destas excelentes propriedades, aliadas aos baixos custos de produção, componentes de ADI tem sido agregados a peças e componentes, notavelmente da indústria automotiva, sendo competitivos com produtos forjados\cite{Hayrynen2002}.

O relativo sucesso do ADI veio com o recente desenvolvimento de ligas ferrosas contendo significativas quantidades de austenita retida estabilizada pelo enriquecimento em carbono. Uma austenita retida suficientemente estável confere elevada ductilidade ao material, enquanto que o microconstituinte ferrítico é responsável por conferir alta resistência mecânica. Originalmente explorada na produção do ADI, recentemente a ideia de microestruturas contendo misturas de ferrita bainítica e austenita retida tem sido explorada para a obtenção de aços de elevada resistência mecânica, como aços TRIP. Nesse contexto, destacam-se os trabalhos de Bhadeshia, Edmonds e Caballero\cite{Bhadeshia1980a,Caballero2005,Garcia-Mateo2009,GarciaMateo2005}. Contrariando o termo ausferrita utilizado em fundição, o agregado de ferrita bainítica e austenita possui diferentes terminologias na literatura, sendo frequentemente rotulado de bainita isenta de carbonetos (do inglês \textit{carbide-free bainite}), ou superbainita.

A ideia de que quantidades substanciais de austenita retida são beneficiais para as propriedades mecânicas também tem sido explorada por diferentes rotas de tratamento térmico. \citaremsentenca{Speer2003} propuseram uma nova rota de tratamento térmico denominada \textit{Quenching and Partitioning}, ou ``Têmpera e Partição'' (doravante T\&P), para obtenção de uma microestrutura constituída de uma mistura de martensita em equilíbrio metaestável com austenita. O conceito deste processo envolve a têmpera parcial da austenita em temperaturas intermediárias às temperaturas de início e fim da transformação martensítica (Ms e Mf, respectivamente), seguido de um tratamento térmico denominado de partição, no qual é promovida a redistribuição do carbono da martensita para a austenita. Uma austenita com mais elementos de liga é menos propensa a se transformar em martensita durante o resfriamento, devido ao abaixamento da temperatura Ms. Dessa forma, a partição de carbono para a austenita visa a estabilização desta fase à temperatura ambiente. \citaremsentenca{Speer2003} propuseram que este processo somente é possível caso as reações de revenimento da martensita, dentre as quais a precipitação de carbonetos é a principal, sejam suprimidas para que haja o enriquecimento da austenita em carbono. Isto seria conseguido pela adição de elementos supressores da formação de cementita, como o Si e o Al.

Speer também sugeriu que o processo T\&P poderia ser utilizado na substituição do processo de produção do ADI. Os resultados dos poucos trabalhos publicados sobre este assunto se mostraram promissores, embora difíceis de serem interpretados\cite{Speer2004}. Resultados recentes de trabalhos conduzidos pelo grupo de pesquisa em que se insere o presente projeto sugeriram que a microssegregação inerente aos ferros fundidos poderia justificar a dificuldade reportada. Contudo, estas evidências, de significativa importância científica e tecnológica, carecem de caracterização e análise minuciosa\cite{Silva2013}.

Face a este desafio, este trabalho busca a melhor compreensão dos fenômenos de transformações de fases que envolvem o processo de Têmpera e Partição para o especifico caso de uma liga de ferro fundido nodular. Isto é justificado pelo extenso conhecimento --- tanto científico, quanto tecnológico --- existente sobre o ferro fundido nodular austemperado (ADI), um material que se beneficia de uma mesma mistura de austenita estabilizada por carbono e ferrita acicular, obtida neste caso pela interrupção da reação bainítica. Neste sentido, o ferro fundido nodular temperado e particionado tem o potencial de substituir o ADI em uma quantidade de aplicações industriais.
