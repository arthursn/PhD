\chapter{Objetivos}

% O presente trabalho faz parte de um projeto liderado pelo professor Hélio Goldenstein, com apoio da Tupy Fundições, com o objetivo estudar a aplicação do tratamento térmico de Têmpera e Partição a ferros fundidos. Como frutos deste projeto foram geradas uma dissertação de mestrado de autoria de Anderson J. S. T. da Silva\cite{Silva2013}, que procurou explorar a viabilidade do processo T\&P aplicado a ferros fundidos nodulares, e duas teses de doutorado, uma de autoria de André C. Melado\cite{Melado2018}, com ênfase na caracterização mecânica do ferro fundido temperado e particionado, e a presente tese
O presente trabalho tem como objetivo compreender as transformações de fases, com ênfase na evolução microestrutural e a cinética dos fenômenos, durante e após a aplicação do tratamento térmico de Têmpera e Partição em um ferro fundido nodular. Derivam do objetivo principal:

\begin{enumerate}
	\item Avaliar o efeito das variáveis de tratamento térmico na microestrutura final;

	\item Caracterizar a cinética da redistribuição de carbono durante a etapa de partição;

	\item Identificar e caracterizar as reações competitivas que podem ocorrer durante o tratamento de partição, como a reação bainítica e a precipitação de carbonetos;

	\item Avaliar o efeito da microssegregação de elementos de liga, inerente a produtos fundidos, na microestrutura final;

	\item Desenvolver um modelo para a cinética local de redistribuição de carbono durante a etapa de partição, considerando o efeito das reações competitivas.
\end{enumerate}
