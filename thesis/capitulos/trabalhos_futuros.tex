\chapter{Sugestões de trabalhos futuros}

\begin{itemize}
	\item Aplicar o conceito de simulação considerando a ocorrência simultânea de precipitação de carbonetos (modelo $CCE\theta$) e reação bainítica durante o processo T\&P para geometrias de simulação 2D e 3D. O modelo utilizado neste trabalho --- \textit{sharp interface} com discretização pelo método de diferenças finitas --- é difícil de ser extrapolado para geometrias 2D e 3D e talvez o método de campo de fases (\textit{Phase Field Method}) seja mais apropriado para este propósito. A aplicação deste modelo não é limitada a ferros fundidos e poderia também ser utilizada para aços.

	\item A partir da implementação do modelo 2D ou 3D, considerar a segregação de elementos substitucionais e outros parâmetros microestruturais herdados da solidificação, como a distribuição e espaçamento dos nódulos de grafita.

	\item Avaliar o efeito de diferentes temperaturas de austenitização durante T\&P de ferros fundidos. Temperaturas maiores implicam em maior teor de carbono na austenita, mas também causam a diminuição da segregação de elementos substitucionais, dessa forma afetando a distribuição de martensita no material.

	\item De forma semelhante ao item anterior, sugere-se avaliar a utilização de temperaturas de austenitização intercríticas, ou seja, no campo de equilíbrio ferrita + austenita + grafita. Como a ferrita tende a formar-se em regiões com menores concentrações de elementos gamagênicos, é possível que o efeito de segregação na transformações de fases seja diminuído. Além disso, microestruturas de ferrita devem possuir menor resistência e maior ductilidade, gerando materiais com propriedades mais semelhantes aos ADIs convencionais.

	\item Explorar ligas contendo Ni e Mo, de modo a diminuir a temperatura Bs e atrasar a reação bainítica. Com isso, abre-se a possibilidade de se avaliar microestruturas T\&P sem a presença de bainita. No entanto, é provável que carbonetos precipitados na martensita continuem presentes.

	% \item Disponibilidade linha Jatobá Sirius
	% Gleeble XTMS alta energia. 3DXRD?
\end{itemize}

% sugestóes de estudos futuros - cadê?
% - considerar a segrega;ao de substitucionais da solidifica;áo e rever o modelamento.
% - fazer isto para diferentes condi;óes de solidifica;aó tais como velocidade de resfriamento e inocula;áo.
% - inoculantes que resultam em populacao binaria de nodulos - como fica a segrega;áo de substitucionais e como isto se reflete no modelamento apresentado
