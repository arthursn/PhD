\resumo{
  \setlength\parindent{0pt}
  Este trabalho está inserido em um projeto que procura estudar a viabilidade técnica da aplicação de um relativamente novo conceito de tratamento térmico, chamado de Têmpera e Partição (T\&P), como alternativa para o processamento de ferros fundidos nodulares com alta resistência mecânica. O processo T\&P tem por objetivo a obtenção de microestruturas multifásicas constituídas de martensita e austenita retida, estabilizada em carbono. A martensita confere elevada resistência mecânica, enquanto a austenita confere ductilidade. No processo T\&P, após a austenitização total ou parcial da liga, o material é temperado até uma temperatura de têmpera $T_T$ entre as temperaturas Ms e Mf para produzir uma mistura controlada de martensita e austenita. Em seguida, na etapa de partição, o material é mantido isotermicamente em uma temperatura igual ou mais elevada (denominada temperatura de partição $T_P$) para permitir a partição de carbono da martensita para a austenita. O carbono em solução sólida diminui a temperatura Ms da austenita, estabilizando-a à temperatura ambiente. O presente trabalho procurou estudar aspectos de transformações de fases --- com ênfase na evolução microestrutural e cinética das reações --- do tratamento térmico de Têmpera e Partição (T\&P) aplicado a uma liga de ferro fundido nodular (Fe--3,47\%C--2,47\%Si--0,2\%Mn). Tratamentos térmicos consistiram de austenitização a \SI{880}{\degreeCelsius} por 30~min, seguido de têmpera a 140, 170 e \SI{200}{\degreeCelsius} e partição a 300, 375 e \SI{450}{\degreeCelsius} por até 2~h. A caracterização microestrutural foi feita por microscopia óptica (MO), eletrônica de varredura (MEV), difração de elétrons retroespalhados (EBSD) e análise de microssonda eletrônica (EPMA). A análise cinética foi feita por meio de ensaios de dilatometria de alta resolução e difração de raios X in situ usando radiação síncrotron. Resultados mostram que a ocorrência de reações competitivas --- reação bainítica e precipitação de carbonetos na martensita --- é inevitável durante a aplicação do tratamento T\&P à presente liga de ferro fundido nodular. A cinética da reação bainítica é acelerada pela presença da martensita formada na etapa de têmpera. A reação bainítica acontece, a baixas temperaturas, desacompanhada da precipitação de carbonetos e contribui para o enriquecimento em carbono, e consequente estabilização, da austenita. Devido à precipitação de carbonetos na martensita, a formação de ferrita bainítica é o principal mecanismo de enriquecimento em carbono da austenita. A microssegregação proveniente da etapa de solidificação permanece no material tratado termicamente e afeta a distribuição da martensita formada na etapa de têmpera e a cinética da reação bainítica. Em regiões correspondentes a contornos de célula eutética são observadas menores quantidades de martensita e a reação bainítica é mais lenta. A microestrutura final produzida pelo tratamento T\&P aplicado ao ferro fundido consiste de martensita revenida com carbonetos, ferrita banítica e austenita enriquecida estabilizada pelo carbono. Adicionalmente, foi desenvolvido um modelo computacional que calcula a redistribuição local de carbono durante a etapa de partição do tratamento T\&P, assumindo os efeitos da precipitação de carbonetos e a ocorrência do crescimento de placas de ferrita bainítica a partir da austenita. O modelo mostrou que a cinética de partição de carbono da martensita para a austenita é mais lenta quando os carbonetos precipitados são mais estáveis e que, quando a energia livre dos carbonetos é suficientemente baixa, o fluxo de carbono acontece da austenita para a martensita. A aplicação do modelo não se limita às condições estudadas neste trabalho e pode ser aplicada para o planejamento de tratamentos T\&P para aços.
}
\palavrachave{Têmpera e Partição}
\palavrachave{Ferro fundido nodular}
\palavrachave{Tratamentos térmicos}
\palavrachave{Martensita}
\palavrachave{Bainita}
\palavrachave{Austenita}
\palavrachave{Modelo computacional}

\resumole{
  \setlength\parindent{0pt}

  The present work belongs to a bigger project whose main goal is to study the technical feasibility of the application of a relatively new heat treating concept, called Quenching and Partitioning (Q\&P), as an alternative to the processing of high strength ductile cast irons. The aim of the Q\&P process is to obtain multiphase microstructures consisting of martensite and carbon enriched retained austenite. Martensite confers high strength, whereas austenite confers ductility. In the Q\&P process, after total or partial austenitization of the alloy, the material is quenched in a quenching temperature $T_Q$ between the Ms and Mf temperatures to produce a controlled mixture of martensite and austenite. Next, at the partitioning step, the material is isothermally held at a either equal or higher temperature (so called partitioning temperature $T_P$) in order to promote the carbon diffusion (partitioning) from martensite to austenite. The present work focus on the study of phase transformations aspects --- with emphasis on the microstructural evolution and kinetics of the reactions --- of the Q\&P process applied to a ductile cast iron alloy (Fe--3,47\%C--2,47\%Si--0,2\%Mn). Heat treatments consisted of austenitization at \SI{880}{\degreeCelsius} for 30~min, followed by quenching at 140, 170, and \SI{200}{\degreeCelsius} and partitioning at 300, 375 e \SI{450}{\degreeCelsius} up to 2~h. The microstructural characterization was carried out by optical microscopy (OM), scanning electron microscopy (SEM), electron backscattered diffraction (EBSD), and electron probe microanalysis (EPMA). The kinetic analysis was studied by high resolution dilatometry tests and in situ X-ray diffraction using a synchrotron light source. Results showed that competitive reactions --- bainite reaction and carbides precipitation in martensite --- is unavoidable during the Q\&P process. The bainite reaction kinetics is accelerated by the presence of martensite formed in the quenching step. The bainite reaction occurs at low temperatures without carbides precipitation and contributes to the carbon enrichment of austenite and its stabilization. Due to carbides precipitation in martensite, growth of bainitic ferrite is the main mechanism of carbon enrichment of austenite. Microsegregation inherited from the casting process is present in the heat treated material and affects the martensite distribution and the kinetics of the bainite reaction. In regions corresponding to eutectic cell boundaries less martensite is observed and the kinetics of bainite reaction is slower. The final microestructure produced by the Q\&P process applied to the ductile cast iron consists of tempered martensite with carbides, bainitic ferrite, and carbon enriched austenite. Additionally, a computational model was developed to calculate the local kinetics of carbon redistribution during the partitioning step, considering the effects of carbides precipitation and bainite reaction. The model showed that the kinetics of carbon partitioning from martensite to austenite is slower when the tempering carbides are more stable and that, when the carbides free energy is sufficiently low, the carbon diffuses from austenite to martensite. The model is not limited to the studied conditions and can be applied to the development of Q\&P heat treatments to steels.
}
\palavrachavele{Quenching and Partitioning}
\palavrachavele{Ductile cast iron}
\palavrachavele{Heat treatments}
\palavrachavele{Martensite}
\palavrachavele{Bainite}
\palavrachavele{Austenite}
\palavrachavele{Computational model}

%%%%%%%%%%%%%

\agradecimentos{
  \setlength\parindent{0cm}
  \setlength{\parskip}{1em}

  % Muitos evento chave em minha vida me levaram a este momento, em .

  Ao Professor Hélio Goldenstein, meu pai acadêmico, não somente pela orientação e amizade ao longo dos vários anos em que me acolheu em seu grupo, mas também por ter compartilhado comigo de sua visão de mundo. Certamente sou um ser humano melhor por ter passado dez anos de minha vida sob sua influência.

  Aos vários co-orientadores com quem tive a boa sorte de ter trabalhado durante este projeto de doutorado. A começar pelo Professor André Paulo Tschiptschin, oficialmente co-orientador desta tese, com quem pude contribuir em diferentes projetos e que me forneceu valiosa análise crítica dos resultados apresentados neste texto. À Professora Maria Jesus Santofimia Navarro e ao Professor Jilt Sietsma, que me receberam na Universidade Tecnológica de Delft e me estimularam a desenvolver o modelo computacional apresentado neste texto. Aos professores Tadashi Fuhuhara e Goro Miyamoto, do Instituto para Pesquisa de Materiais da Universidade de Tohoku, que me receberam em seu grupo e com quem tive valiosas discussões sobre os resultados experimentais e, em particular, sobre a utilização da técnica de EBSD.

  Aos meus pais, Taeko e Iwao, por acreditarem que a educação de seus filhos é o maior legado que eles poderiam nos deixar.

  À Tupy Fundições e, em especial, ao Anderson José Saretta Tomaz da Silva e ao Professor Wilson Luiz Guesser, pela fundição e fornecimento da liga de ferro fundido estudada neste trabalho.
  
  Ao Leonardo Wu, pela grande ajuda durante os experimentos na estação XTMS no LNNano/LNLS.

  Agradeço ao meu grande amigo Edwan Anderson Ariza Echeverri, pela constante amizade, discussões e constante apoio, além de sua valiosa ajuda durante os longos turnos de experimentos no LNLS.
  
  % Aos técnicos Leandro Moraes e Marcos Mansueto, do Laboratório de Microssonda Eletrônica do Instituto de Geociências da USP, pela ajuda na realização das análises de EPMA.

  Fica meu agradecimento ao Caio Fattori e, por extensão, à toda comunidade de software livre, pelo desenvolvimento da classe de documento LaTeX USPTeX, sem a qual o trabalho de escrita desta tese seria imensamente mais tortuoso.

  Aos funcionários do PMT-USP e, especialmente, à Suellen Alves Nappi, pela ajuda ao longo dos vários anos em que estive no departamento.

  A todos amigos do PMT-USP e do Laboratório de Transformações de Fases, pela ajuda, discussões, companheirismo e por tornarem o ambiente de pesquisa tão acolhedor e agradável.

  À Capes, pela concessão da bolsa de doutorado e da bolsa de doutorado sanduíche no exterior (projeto PDSE 7409/2015-00).

  Ao Centro de Colaboração Internacional do Instituto para Pesquisa de Materiais da Universidade de Tohoku (ICC-IMR), pela concessão da bolsa para realização de estágio de pesquisa na Universidade de Tohoku.
}

%%%%%%%%%%%%%

\assunto{Têmpera e Partição}
\assunto{Ferro fundido nodular}
\assunto{Tratamentos térmicos}
\assunto{Modelo computacional}
\assunto{Martensita}

\edicaocorrigida
\FCautorresumido{Nishikawa, A. S.}
\FCautorexpandido{Nishikawa, Arthur Seiji}
