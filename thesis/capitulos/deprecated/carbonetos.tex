\subsection{Caracterização dos carbonetos precipitados durante T\&P}

A caracterização microestrutural das amostras temperadas e particionadas mostra a presença de carbonetos de revenimento na martensita, mesmo para os tempos mais curtos de partição, indicando que a precipitação de carbonetos ocorre rapidamente durante a etapa de aquecimento da temperatura de têmpera $T_T$ até a temperatura de partição $T_P$. No entanto, a natureza dos carbonetos precipitados, embora possível de ser inferida pelas temperaturas de tratamento, carece de evidência experimental. Nesta seção, os carbonetos precipitados durante o processo T\&P são caracterizados por técnicas de difração.

Experimentos de difração de raios X utilizando fonte síncrotron foram conduzidos nas seguintes condições:

\begin{itemize}

\end{itemize}

% Experimentos de difração de raios X utilizando fonte síncrotron foram conduzidos na estação experimental XMTS no Laboratório Nacional de Luz Síncrotron. Na instalação XTMS são disponibilizados diferentes conjuntos de detectores que permitem tanto a avaliação em tempo real de frações e parâmetros cristalográficos de fases, como também aquisições com ótimos níveis sinal/ruído. No presente trabalho, detetores 1D Mythen-1K foram utilizados para a análise \enfase{in situ} das transformações de fases durante a etapa de partição (mostrada adiante na seção \ref{sec:DRXInSitu}), enquanto o detetor 2D Rayonix SX165 foi utilizado para aquisições \enfase{ex situ} de amostras previamente tratadas termicamente pelo processo T\&P. Tempos longos de aquisição a obtenção de difratogramas com excelentes níveis de sinal/ruído, 