\chapter{Próximas etapas}

Embora a formação de bainita tenha sido identificada como um mecanismo de partição de carbono para a austenita, ainda não está claro o papel da martensita neste fenômeno. A martensita não aparenta particionar carbono para a austenita de forma significante, como esperado, e as hipóteses formuladas para explicar esse fenômeno estão associadas à possível ocorrência de reações de revenimento nesta fase. Ou seja, de que há a ocorrência de segregação do carbono para os defeitos cristalinos da martensita (discordância e maclas) e precipitação de carbonetos de transição durante a etapa de partição. No entanto, à exceção da amostra particionada a 450 °C, não há evidências conclusivas dos dois fenômenos pelas técnicas utilizadas até o momento.

Dessa forma, as próximas etapas do trabalho devem se concentrar em estudar o revenimento da martensita no ferro fundido via medidas globais, como dilatometria e difração de raios X, e realizar uma melhor caracterização microestrutural das placas de martensita após o processo T\&P. Dada a escala nanométrica dos constituintes a serem identificados, essa caracterização deverá ser feita por meio de Microscopia Eletrônica de Transmissão (MET) e/ou outra técnica com capacidade semelhante de resolução espacial (e.g., Tomografia de Sonda Atômica - 3D-APT).

Além disso, é interessante compreender melhor a distribuição da austenita ao longo do material tratado termicamente. Imagens de MO e MEV permitem caracterizar bem a austenita poligonal na forma de blocos, localizada entre feixes de bainita e placas de martensita. Entretanto, a caracterização de filmes de austenita presentes entre as ripas de ferrita bainítica é deveras complicada por meio destas técnicas, muito em decorrência do ataque metalográfico, que gera um efeito de relevo pronunciado e prejudicial para interpretação desse tipo de microconstituinte. É planejado que a técnica de Difração de Elétrons Retroespalhados (EBSD), que abdica desse ataque para análise e permite identificar localmente a estrutura cristalina do produto, seja utilizada para esse propósito. A preparação adequada da superfície para análise de EBSD, todavia, ainda é uma dificuldade técnica encontrada dentro do grupo de pesquisa.

Estas etapas de caracterização deverão ser realizadas ao longo do primeiro semestre de 2016 durante estágio de doutorado sanduíche na \enfase{Technische Universiteit Delft} (TU Delft), sob cotutela da Profª Maria Santofimia Navarro. O grupo de pesquisa da Profª Santofimia tem gerado vasta produção no assunto do trabalho e, portanto, esta interação é de interesse para desenvolvimento do projeto. Parte das análises também podem ser conduzidas no Brasil, via cooperação com instalações Multiusuário, sendo enfatizado o Laboratório de Microscopia Eletrônica do LNNano (LME-LNNano).

%Por fim, há a possibilidade de análise de algumas amostras por Tomografia de Sonda Atômica (\enfase{3D Atom Probe Tomography} - 3D-APT) nos EUA via submissão de proposta ao \enfase{Oak Ridge National Laboratory}. Recentemente,  criadas no grupo de pesquisa
