\chapter{Conclusões}

O presente trabalho procurou estudar aspectos de transformações de fases --- com ênfase na evolução microestrutural e cinética das reações --- do tratamento térmico de Têmpera e Partição (T\&P) aplicado a uma liga de ferro fundido nodular. Em relação às observações experimentais, as seguintes conclusões foram obtidas:

\begin{enumerate}
  \item A segregação de elementos de liga, proveniente da etapa de solidificação do metal, não foi completamente eliminada durante a austenitização da liga. Regiões de contornos de célula apresentam maiores teores de C e Mn, enquanto que regiões próximas a nódulos de grafita apresentam maiores teores de Si e Cu.

  \item A distribuição da martensita formada durante a etapa de têmpera é afetada pela microssegregação. Regiões junto a contornos de célula eutética apresentam menores quantidades de martensita do que nas regiões próximas a nódulos de grafita. A distribuição da martensita no material se torna mais homogênea com a diminuição da temperatura de têmpera.

  \item A precipitação de carbonetos na martensita acontece durante a etapa de aquecimento desde a etapa de têmpera até a temperatura de partição. Carbonetos de transição do tipo $\eta$ são precipitados nas menores temperaturas de partição (300 e \SI{375}{\degreeCelsius}), enquanto que a cementita é precipitada na maior temperatura (\SI{450}{\degreeCelsius}).

  \item A reação bainítica acontece durante a etapa de partição do tratamento T\&P, sendo observada para todas as condições. Nas menores temperaturas de partição (300 e \SI{375}{\degreeCelsius}) a reação não é acompanhada por quantidades apreciáveis de carbonetos e promove o enriquecimento da austenita em carbono. Para a maior temperatura (\SI{450}{\degreeCelsius}) sob tempos curtos (< 5~min) a reação bainítica também promove o enriquecimento em carbono da austenita. Após 5~min a precipitação de carbonetos acontece, consumindo toda a austenita após 15~min.

  \item A partição de carbono entre a mistura de martensita + carbonetos e a austenita é observada para tempos curtos de partição (30~s) a \SI{375}{\degreeCelsius}.

  \item O carbono rejeitado durante a reação bainítica faz com que a partição de carbono da mistura martensita + carbonetos para a austenita seja limitada aos primeiros instantes da etapa de partição. Assim, a formação de ferrita bainítica é o principal mecanismo de enriquecimento em carbono da austenita.

  \item A diminuição da temperatura de partição tem o efeito de refinar o produto bainítico. 

  \item A diminuição da temperatura de têmpera refina a microestrutura do produto bainítico pela repartição dos grãos de austenita pela maior fração de placas de martensita.

  \item A microestrutura final produzida pelo tratamento T\&P aplicado ao ferro fundido consiste de martensita revenida com carbonetos, ferrita banítica e austenita enriquecida estabilizada pelo carbono.
\end{enumerate}

Adicionalmente, foi desenvolvido um modelo computacional que calcula a cinética da redistribuição local de carbono durante a etapa de partição do tratamento T\&P assumindo os efeitos da precipitação de carbonetos e a ocorrência do crescimento de placas de ferrita bainítica a partir da austenita. O modelo mostrou que a cinética de partição de carbono da martensita para a austenita é mais lenta quando os carbonetos precipitação são mais estáveis e que, quando a energia livre dos carbonetos é suficientemente baixa, o fluxo de carbono acontece da austenita para a martensita.
Quando a reação bainítica é considerada, o modelo prevê diferentes cenários dependendo da energia livre dos carbonetos. Carbonetos com alta energia livre são dissolvidos rapidamente e causam rápida partição de carbono da martensita + carbonetos para a austenita. Carbonetos mais estáveis dissolvem-se para tempos curtos de partição, mas na medida que o carbono rejeitado durante o crescimento da ferrita bainítica interage como o carbono proveniente da martensita, o fluxo de carbono é revertido. 


% O modelo foi aplicado para as condições encontradas experimentalmente neste trabalho e foi utilizado para explicar porque a não dissolução dos carbonetos de transição na martensita não ocorre quando 



  % Para tempos curtos de partição a reação não é acompanhada pela precipitação de carbonetos, sendo apenas a ferrita bainítica observada. 

  % \item A formação de ferrita bainítica durante a etapa de partição foi observada em todas as condições de tratamento térmico e provou ser um mecanismo de enriquecimento em carbono da austenita durante o processo T\&P. Não foi possível separar a contribuição da formação de ferrita bainítica da partição de carbono da martensita para a austenita nas curvas de redistribuição de carbono. Contudo, o comportamento semelhante das curvas de transformação e enriquecimento de carbono sugere fortemente que a reação bainítica é a principal responsável pela redistribuição de carbono durante T\&P e consequente estabilização da austenita após o tratamento.
  
  % \item Evidência metalográfica de precipitação de carbonetos foi obtida apenas para as amostras particionadas a 450 °C, nas quais foram estes precipitados foram observados na forma de dispersões muito finas. No entanto, resultados em fase de análise, não mostrados nesta versão do texto, sugerem que a precipitação de carbonetos --- provavelmente de transição --- também pode ter ocorrido nas menores temperaturas de partição.

  % \item As quantidades de austenita obtidas nas condições estudadas foram consistentemente menores do que as previstas pelo modelo de Equilíbrio Restringido de Carbono. Supõe-se que isso é ocorre porque parte do carbono disponível persiste preso a defeitos (supersaturação) ou na forma de precipitados (carbonetos). Isso diminui a quantidade de carbono livre para ser particionado para a austenita.
  
  % \item A segregação de elementos de liga desempenha papel na localização das maiores concentrações de austenita retida ao longo do material. Regiões correspondentes a antigos contornos de célula eutética, por possuírem maiores quantidades de elementos de liga, apresentaram maior concentração de austenita.

  % \item As microestruturas finais obtidas consistiram de martensita revenida, ferrita banítica e austenita enriquecida estabiliza pelo carbono. Estas microestruturas multifásicas puderam ser adaptadas variando três principais variáveis de tratamento térmico: a temperatura de têmpera e a temperatura e o tempo de partição.

  % \item A temperatura de têmpera controla as quantidades de martensita e a escala do produto bainítico formado durante a partição por um mecanismo de repartição dos grãos de austenita. Em menores temperaturas de têmpera foram obtidas microconstituintes bainíticos mais refinados.

  

  

  %\item A partição de carbono da martensita em placas para a austenita é mínima, ou não chega a ocorrer. Foram obtidas evidências de que após o ciclo T\&P ou a martensita prevalece supersaturada em carbono, ou a supersaturação é eliminada pela precipitação de carbonetos de revenimento. No entanto, apenas nas amostras particionadas a 450 °C foi constatada intensa formação de carbonetos.
  %\item Com exceção das amostras temperadas a 200 °C, as quantidades de austenita obtidas nas condições estudadas foram menores do que as previstas pelo modelo de Equilíbrio Restringido de Carbono.
  %\item A cinética da reação bainítica durante a etapa de partição é afetada pela quantidade de martensita formada durante a etapa de têmpera. Menores temperaturas de têmpera levam ao enriquecimento mais rápido da austenita. Propõe-se que isso resulta de um efeito catalizador provocado pela presença da martensita.
  %\item A temperatura de partição exerce forte efeito na cinética e na escala do produto formado na reação bainítica. Temperaturas de partição menores originam produtos mais refinados associados a uma cinética mais lenta de transformação.