\chapter{Conclusões parciais}

As seguintes conclusões podem ser tiradas do trabalho:

\begin{enumerate}
	\item A reação bainítica se mostrou o principal mecanismo de enriquecimento de carbono na austenita durante o processo T\&P aplicado à liga estudada. A temperatura de partição de 200 °C foi a única na qual não foi observado enriquecimento em carbono da austenita, embora o ciclo térmico tenha afetado a temperatura Ms da austenita  durante o resfriamento final.
	\item A partição de carbono da martensita em placas para a austenita é mínima, ou não chega a ocorrer. Foram obtidas evidências de que após o ciclo T\&P ou a martensita prevalece supersaturada em carbono, ou a supersaturação é eliminada pela precipitação de carbonetos de revenimento. No entanto, apenas nas amostras particionadas a 450 °C foi constatada intensa formação de carbonetos.
	\item Com exceção das amostras temperadas a 200 °C, as quantidades de austenita obtidas nas condições estudadas foram menores do que as previstas pelo modelo de Equilíbrio Restringido de Carbono.
	\item A cinética da reação bainítica durante a etapa de partição é afetada pela quantidade de martensita formada durante a etapa de têmpera. Menores temperaturas de têmpera levam ao enriquecimento mais rápido da austenita. Propõe-se que isso resulta de um efeito catalizador provocado pela presença da martensita.
	\item A temperatura de partição exerce forte efeito na cinética e na escala do produto formado na reação bainítica. Temperaturas de partição menores originam produtos mais refinados associados a uma cinética mais lenta de transformação.
\end{enumerate}