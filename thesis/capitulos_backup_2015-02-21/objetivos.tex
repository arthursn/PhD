\chapter{Objetivos}

Neste trabalho um ferro fundido nodular foi tratado termicamente segundo a rota de Têmpera e Partição. Isto é justificado pelo extenso conhecimento --- tanto científico, quanto tecnológico --- existente sobre o ferro fundido nodular austemperado (ADI), um material que se beneficia de uma mesma mistura de austenita estabilizada por carbono e ferrita acicular, obtida neste caso pela interrupção da reação bainítica. Neste sentido, o ferro fundido nodular temperado e particionado tem o potencial de substituir o ADI em uma quantidade de aplicações industriais.

O presente trabalho tem como objetivo caracterizar os microconstituintes presentes em um ferro fundido nodular submetida ao processo de Têmpera e Partição. Derivam do objetivo principal:

\begin{enumerate}
\item O estudo do efeito da intensa segregação presente em ferros fundidos na microestrutura final; 
\item A identificação de reações competitivas durante o tratamento de partição, como a reação bainítica e a precipitação de carbonetos;
%\item A investigação da ocorrência de transformação martensítica isotérmica em tratamentos próximos à temperatura Ms; 
\end{enumerate}
