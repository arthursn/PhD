\section{Conclusões}

\begin{frame}{Conclusões}
  % O presente trabalho procurou estudar aspectos de transformações de fases --- com ênfase na evolução microestrutural e cinética das reações --- do tratamento térmico de Têmpera e Partição (T\&P) aplicado a uma liga de ferro fundido nodular. Em relação às observações experimentais, as seguintes conclusões foram obtidas:

  \begin{enumerate}
    \item \textbf{Microssegregação} não foi completamente eliminada durante a austenitização e causa uma \textbf{distribuição heterogênea da martensita}. Contornos de célula apresentam mais martensita do que regiões próximas a nódulos de grafita.

    \item Precipitação de \textbf{carbonetos na martensita} acontece durante o aquecimento desde a temperatura de têmpera até a temperatura de partição. Carbonetos de transição ($\eta$ ou $\epsilon$) são precipitados a 300 e a \SI{375}{\degreeCelsius}. Cementita precipita a \SI{450}{\degreeCelsius}.

    \item A \textbf{reação bainítica} acontece durante a etapa de partição. Para $T_P$ igual a 300 e \SI{375}{\degreeCelsius}) a reação ocorre sem precipitação de carbonetos (ferrita bainítica), promovendo enriquecimento da austenita em carbono. A \SI{450}{\degreeCelsius} sob tempos curtos (< 5~min) a reação bainítica também promove o enriquecimento em carbono da austenita. Para tempos longos a precipitação de carbonetos acontece, consumindo toda a austenita.

    \item A reação bainítica é acelerada na presença de martensita devido à nucleação de bainita nas interfaces $\alpha'/\gamma$

    \seti
  \end{enumerate}
\end{frame}

\begin{frame}{Conclusões}
  \begin{enumerate}
    \conti

    \item A partição de carbono entre a $\alpha' + \theta$ e $\gamma$ é observada para tempos curtos de partição (30~s) a \SI{375}{\degreeCelsius}.

    \item O carbono rejeitado durante a reação bainítica faz com que a partição de carbono entre $\alpha' + \theta$ e $\gamma$ seja limitada aos primeiros instantes da etapa de partição. Assim, a formação de \textbf{ferrita bainítica é o principal mecanismo de enriquecimento em carbono da austenita}.

    \item A diminuição da \textbf{temperatura de partição} tem o efeito de refinar o produto bainítico. A diminuição da \textbf{temperatura de têmpera} refina a microestrutura do produto bainítico pela repartição dos grãos de austenita pela maior fração de placas de martensita.

    \item A microestrutura final produzida pelo tratamento T\&P aplicado ao ferro fundido consiste de martensita revenida com carbonetos, ferrita banítica e austenita enriquecida estabilizada pelo carbono.
  \end{enumerate}
\end{frame}

  % Adicionalmente, foi desenvolvido um modelo computacional que calcula a cinética da redistribuição local de carbono durante a etapa de partição do tratamento T\&P assumindo os efeitos da precipitação de carbonetos e a ocorrência do crescimento de placas de ferrita bainítica a partir da austenita. O modelo mostrou que a cinética de partição de carbono da martensita para a austenita é mais lenta quando os carbonetos precipitação são mais estáveis e que, quando a energia livre dos carbonetos é suficientemente baixa, o fluxo de carbono acontece da austenita para a martensita.
  % Quando a reação bainítica é considerada, o modelo prevê diferentes cenários dependendo da energia livre dos carbonetos. Carbonetos com alta energia livre são dissolvidos rapidamente e causam rápida partição de carbono da martensita + carbonetos para a austenita. Carbonetos mais estáveis dissolvem-se para tempos curtos de partição, mas na medida que o carbono rejeitado durante o crescimento da ferrita bainítica interage como o carbono proveniente da martensita, o fluxo de carbono é revertido. 

